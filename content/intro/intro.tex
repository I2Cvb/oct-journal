% include the figures path relative to the master file
\graphicspath{ {./content/intro/figures/} }

\section{Introduction}

Eye diseases such as \ac{dr} and \ac{dme} are the most common causes of irreversible vision loss in individuals with diabetes.
Just in United States alone, health care and associated costs related to eye diseases are estimated at almost \SI{500}[\$]{M}~\cite{Sharma2005}.
%It is estimated that eye diseases will cost US\$500 million annually in healthcare and associated costs in the United States alone~\cite{Sharma2005}.
Moreover, the prevalent cases of \ac{dr} are expected to grow exponentially affecting over \SI{300}{M} people worldwide by 2025~\cite{Wild2004}.
%Moreover, the prevalence of \ac{dr} is expected to grow exponentially and affect over 300 millions people worldwide by 2025~\cite{Wild2004}.
Early detection and treatment of \ac{dr} and \ac{dme} play a major role to prevent adverse effects such as blindness.
Indeed, the detection and diagnosis of retinal diseases are based on the detection of vascular abnormalities or lesions in the retina. 

In past decades, \ac{cad} systems devoted to ophthalmology, have been developed focusing on the automatic analysis of fundus images~\cite{Abramoff2010,Trucco2013}.
%\Ac{cad} systems have focused on the automatic analysis of fundus images in past decades~\cite{Abramoff2010,Trucco2013}.
However, the use of fundus photography is limited to the detection of signs which are correlated with retinal thickening such as hard and soft exudates, hemorrhages or micro-aneurysms.
However, \ac{dme} is characterized as an increase in retinal thickness within 1 disk diameter of the fovea center with or without hard exudates and sometimes associated with cysts~\cite{ETDRSG1985}.
Therefore, fundus photography cannot always identify the clinical signs of \ac{dme}; for example cysts, which are not visible in the retinal surface. In addition, it does not provide any quantitative measurements of retina thickness or information about cross-sectional retinal morphology. 

Recently, \ac{oct} has been widely used as a valuable diagnosis tool for \ac{dme} detection.
\ac{oct} is based on optical reflectivity and produces cross-sectional and three-dimensional images of the central retina, thus allowing quantitative retinal thickness and structure measurements.
The new generation of \ac{oct} imaging, namely \ac{sdoct} offers higher resolution and faster image acquisition over conventional time domain \ac{oct}. \Ac{sdoct} can produce $27,000$ to $40,000$ A-scans/seconds with an axial resolution ranging from \SIrange{3.5}{6}{\micro \metre}~\cite{Chen2005}. 
Many of the previous works on \ac{oct} image analysis have focused on the problem of retinal layers segmentation, which is a necessary step for retinal thickness measurements~\cite{Chiu2010,Kafieh2013}.
Few works have addressed the specific problem of \ac{dme} and its associated features detection from \ac{oct} images. 


In this paper, we propose a method for automatic identification of patients with \ac{dme} versus normal subjects by classifying the \ac{oct} volumes. Our method is based on \ac{lbp} features to describe the texture of \ac{oct} images and dictionary learning using the \ac{bow} models~\cite{Sivic2003}.
However, our method do not rely on keypoints detection as opposed to the work of Venhuizen\,\textit{et al.} who also employed the \ac{bow} models~\cite{Venhuizen2015}. We rather divide the images into local patches and extract a dense set of \ac{lbp} descriptors.
We also use the entire \ac{oct} volume and extract 3D-\ac{lbp} features to describe the volume, which is different from the work of Liu\,\textit{et al.} who classified only the foveal scan for each patient~\cite{Liu2011}.
%We will show in the experiments, Sect.\,\ref{sec:method}, that using the 3D-\ac{lbp} descriptor provides better classification performances than extraction \ac{lbp} in each individual B-scan.

This paper is organized as follows. Section~\ref{sec:method} describes the features extraction methodology and the classification approach based on the \ac{bow} models. Experiments and results are discussed in Sect.\,\ref{sec:exp}. Conclusions and avenue for future directions are drawn in Sect.\,\ref{sec:con}.

%----------

%%% Local Variables:
%%% TeX-master: "../../main.tex"
%%% End:
