\title{Classification of SD-OCT Volumes using Local Binary Patterns: Experimental Validation for DME Detection}
\tnotetext[ghSource]{Document source available in GitHub~\cite{Lemaitre2015}}


%% or include affiliations in footnotes:
\author[le2i]{Guillaume~Lema\^itre\corref{mycorrespondingauthor}}
\ead{g.lemaitre58@gmail.com}
\author[le2i]{Mojdeh~Rastgoo\corref{mycorrespondingauthor}}
\ead{mojdeh.rastgoo@gmail.com}
\author[le2i]{Joan~Massich\corref{mycorrespondingauthor}}
\ead{joan.massich@u-bourgogne.fr}
% \ead[url]{www.elsevier.com}
\cortext[mycorrespondingauthor]{Corresponding author}
\author[seri]{Carol Y. Cheung}
\author[seri]{Tien Y. Wong}
\author[seri]{Ecosse Lamoureux}
\author[seri]{Dan Milea}
\author[le2i,cisir]{Fabrice~M\'eriaudeau}
\author[le2i]{D\'esir\'e~Sidib\'e}

\address[Le2i]{LE2I UMR6306, CNRS, Arts et M\'etiers, Univ. Bourgogne Franche-Comt\'e, 12 rue de la Fonderie, 71200 Le Creusot, France}
\address[cisir]{Center for Intelligent Signal and Imaging Research (CISIR), Electrical \& Electronic Engineering Department, Universiti Teknologi Petronas, 32610 Seri Iskandar, Perak, Malaysia}
\address[seri]{Singapore Eye Research Institute, Singapore National Eye Center, Singapore}


\begin{abstract}
\acresetall  % reset the acronyms from the title (if any)
This paper addresses the problem of automatic classification of \ac{sdoct} data for automatic identification of patients with \ac{dme} versus normal subjects.
\ac{oct} has been a valuable diagnostic tool for \ac{dme}, which is among the most common causes of irreversible vision loss in individuals with diabetes.
Here, a classification framework with five distinctive steps is proposed and we present an extensive study of each step.
Our method considers combination of various pre-processings in conjunction with \ac{lbp} features and different mapping strategies.
Using linear and non-linear classifiers, we tested the developed framework on a balanced cohort of 32 patients.

Experimental results show that the proposed method outperforms the previous studies by achieving a \ac{se} and \ac{sp} of 81.2\% and 93.7\%, respectively. 
Our study concludes that the 3D features and high-level representation of 2D features using patches achieve the best results.
However, the effects of pre-processing is inconsistent with respect to different classifiers and feature configurations. 
\end{abstract}

\begin{keyword}
\acl{dme}, \acl{oct}, \acs{dme}, \acs{oct}, \ac{lbp}
\end{keyword}
