\title{Classification of \acs*{sdoct} Volumes with \acs*{lbp}: Application to \acs*{dme} Detection}
\tnotetext[ghSource]{Document source available in GitHub~\cite{Lemaitre2015}}


%% or include affiliations in footnotes:
\author[le2i,vicorob]{Guillaume~Lema\^itre\corref{mycorrespondingauthor}}
\ead{g.lemaintre58@gmail.com}
\author[le2i,vicorob]{Mojdeh~Rastgoo\corref{mycorrespondingauthor}}
\ead{mojdeh.rastgoo@gmail.com}
\author[le2i]{Joan~Massich\corref{mycorrespondingauthor}}
\ead{joan.massich@u-bourgogne.fr}
% \ead[url]{www.elsevier.com}
\cortext[mycorrespondingauthor]{Corresponding author}

\author[le2i]{Fabrice~M\'eriaudeau}
\author[le2i]{D\'esir\'e~Sidib\'e}

\address[le2i]{ViCOROB, Universitat de Girona, Campus Montilivi, Edifici P4, 17071 Girona, Spain}
\address[vicorob]{LE2I UMR6306, CNRS, Arts et M\'etiers, Univ. Bourgogne Franche-Comt\'e, 12 rue de la Fonderie, 71200 Le Creusot, France}

\begin{abstract}
\acresetall  % reset the acronyms from the title (if any)
This paper addresses the problem of automatic classification of \ac{sdoct} data for automatic identification of patients with \ac{dme} versus normal subjects.
Our method is based on \ac{lbp} features to describe the texture of \ac{oct} images and we compare different \ac{lbp} features extraction approaches to compute a single signature for the whole \ac{oct} volume.
Experimental results with two datasets of respectively 32 and 30 \ac{oct} volumes show that
regardless of using low or high level representations, features derived from \ac{lbp} texture have highly discriminative power.% for the task on hand.

Moreover, the experiments show that the proposed method achieves better classification performances than other recent published works.
\end{abstract}

\begin{keyword}
\acl{dme}, \acl{oct}, \acs{dme}, \acs{oct}, \ac{lbp}
\end{keyword}
