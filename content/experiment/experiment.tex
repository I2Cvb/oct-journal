% % include the figures path relative to the master file
 \graphicspath{ {./content/experiment/figures/} }
\begin{landscape}
  \begin{table}[ht]
\caption{The outline and summary of the performed experiments.}
\medskip
\scriptsize{
\begin{center}
\resizebox{1\linewidth}{!}{
\begin{tabular}{l  c	 c  c  c  c  c  c  c }
\toprule
\\
&  Dataset & Pre-processing & Features & Mapping & Representation & Classification & Validation & Evaluation \\
\cmidrule(l){2-9}
\\
\multirow{3}{*}{Common:} & \multirow{3}{*}{SERI} & \multirow{3}{*}{\ac{nlm}} & \ac{lbp},\ac{lbptop} & \multirow{3}{*}{} & \multirow{3}{*}{}  & & \multirow{3}{*}{\ac{lopocv}}& \multirow{3}{*}{} \\
        &      &          & $S = \{8,16,24\}$ & & & & & \\
        &      &          & $R = \{1,2,3\}$  & & & & & \\\\
\midrule
\\
Experiment\#1:  \\
%\hdashline \noalign{\vskip 3pt}
%\\
\multirow{2}{4cm}{Goal: Evaluation of features, mapping and representation} & \multirow{2}{*}{+ Duke} &\multirow{2}{*}{$\sim$} & \multirow{2}{*}{$\sim$} & \emph{global} & \ac{bow} & \multirow{2}{*}{\ac{rf}} & \multirow{2}{*}{$\sim$} & \multirow{2}{*}{\ac{se}, \ac{sp}, \cite{Venhuizen2015}}\\
&  & & & \emph{local} & Histogram &  & & \\\\
\midrule
\\
Experiment\#2:\\
%\hdashline \noalign{\vskip 3pt}
%\\
\multirow{2}{4cm}{Goal: Finding the optimum number of words}& \multirow{2}{*}{$\sim$} & + \acs{f} & \multirow{2}{*}{$\sim$} & \emph{global} & \multirow{2}{*}{\ac{bow}} & \multirow{2}{*}{\ac{lr}} & \multirow{2}{*}{$\sim$} & \multirow{2}{*}{ \ac{acc}, \ac{f1}}\\
  & & + \acs{fal} & & \emph{local} & & & & \\\\
\midrule
\\
Experiment\#3: \\
%\hdashline \noalign{\vskip 3pt}
%\\
%\hdashline \noalign{\vskip 3pt}
 \multirow{4}{4cm}{Goal: Evaluation of different pre-processing for high-level features }& \multirow{4}{*}{$\sim$} & \multirow{2}{*}{+\acs{f}} & \multirow{4}{*}{$\sim$} & \multirow{2}{*}{\emph{global}} & \multirow{4}{*}{\ac{bow}} & $3$-\ac{nn} & \multirow{4}{*}{$\sim$} & \multirow{4}{*}{\ac{se}, \ac{sp}}\\
 & & \multirow{2}{*}{+\acs{fal}} & & \multirow{2}{*}{\emph{local}} &  & \ac{rf} & & \\
 & & & & & & \ac{svm} & & \\
 & & & & & & \ac{gb} & & \\
\midrule
\\
Experiment\#4:\\
%\hdashline \noalign{\vskip 3pt}
%\\
%\hdashline \noalign{\vskip 3pt}
\multirow{4}{4cm}{Goal: Evaluation of different pre-processing for low-level features} & \multirow{4}{*}{$\sim$} & \multirow{2}{*}{ +\acs{f}} & \multirow{4}{*}{$\sim$} & \multirow{4}{*}{\emph{global}} & \multirow{4}{*}{Histogram} & $3$-\ac{nn} & \multirow{4}{*}{$\sim$} & \multirow{4}{*}{\ac{se}, \ac{sp}}\\
& & \multirow{2}{*}{+\ac{fal}} & & & & \ac{rf} &  &\\
& & & & & & \ac{svm} & & \\
& & & & & & \ac{gb} & & \\
\\
\bottomrule


\end{tabular}}
\end{center}}
\label{tab:table4}
\end{table}
\end{landscape}

\section{Experiments}
\label{sec:exp} 
A set of three experiments is designed to test the influence of the different blocks of the proposed framework in comparison to our previous work~\cite{Lemaintre2015miccaiOCT}.
These experiments are designed such as: 
\begin{enumerate}
\item[(i)] \emph{Experiment~\#1} evaluates the effects of number of words used in \ac{bow} (high-level representation).
\item[(ii)] \emph{Experiment~\#2} evaluates the effects of different pre-processing steps and classifiers on high-level representation.
\item[(iii)] \emph{Experiment~\#3} evaluates the effects of different pre-processing steps and classifiers on low-level representation.
\end{enumerate}
Table~\ref{tab:experiment_summary} reports the experiments which have been carried out in \cite{Lemaintre2015miccaiOCT} as a baseline and outlines the complementary experimentation here proposed.
The reminder of this section details the common configuration parameters across the experiments, while the detailed explanations are presented in the following 
subsections. 

All the experiments are performed using a private dataset (see Sect.\,\ref{sec:exp:dataset:seri}) and are reported as presented in Sect.\,\ref{sec:exp:validation}.
In all the experiments, \ac{lbp} and \ac{lbptop} features are extracted using both \emph{local} and \emph{global}-mapping for different sampling points of 8, 16, and 24 for radius of 1, 2, and 3 pixels, respectively.
The partitioning for \emph{local}-mapping is set to ($7 \times 7$) pixels patch for 2D \ac{lbp} and ($ 7 \times 7 \times 7$) pixels sub-volume for \ac{lbptop}.

\subsection{SERI-Dataset}\label{sec:exp:dataset:seri}
This dataset was acquired by the Singapore Eye Research Institute (SERI), using CIRRUS TM (Carl Zeiss Meditec, Inc., Dublin, CA) \ac{sdoct} device. The dataset consists of 32 \ac{oct} volumes (16 \ac{dme} and 16 normal cases). Each volume contains 128 B-scan with resolution of 512 $\times$ 1024 pixels.
All \ac{sdoct} images are read and assessed by trained graders and identified as normal or \ac{dme} cases based on evaluation of retinal thickening, hard exudates, intraretinal cystoid space formation and subretinal fluid.

\subsection{Validation}\label{sec:exp:validation}

\input{./content/experiment/figures/evaluation_corolary_def.tex}
\begin{figure}

  \def\myRadius{.65cm}
  \def\vennSpace{(0,0) rectangle (2.6cm,1.6cm)}
  \def\predictedCircle{(.8cm,.8cm) circle (\myRadius)}
  \def\actualCircle{(1.8cm,.8cm) circle (\myRadius)}
  \def\myLabelRadius{.450cm}

  \subfigure[][Confusion matrix with truly and falsely positive samples detected (TP, FP) in the first row, from left to right and the falsely and truly negative samples detected (FN, TN) in the second row, from left to right.]{
    \label{fig:evaluation:confusion_matrix}
    \begin{tikzpicture}[scale=0.5]
      \node at (0,0){
        \begin{tabular}{
            >{\centering}m{1em} >{\centering}m{1em} >{\centering}m{1in} >{\centering\arraybackslash}m{1in}}
          & & \multicolumn{2}{c}{ Actual Class }\\
          & & A+ & A- \\
          \multirow{3}{*}{\rotatebox[origin=c]{90}{Predicted Class}}& P+ &  \tikz{\tp} & \tikz{\fp} \\
          & P- & \tikz{\fn} & \tikz{\tn}
        \end{tabular}
      };
    \end{tikzpicture}
  }\\
  \centering
  \subfigure[][\ac{se} and \ac{sp} evaluation, corresponding to the ratio of the doted area over the blue area.]{
    \label{fig:evaluation:roc_axis}
    \begin{tikzpicture}[scale=0.5]
      \def\seEquation{$SE = \frac{TP}{TP+FN}$}
      \def\spEquation{$SP = \frac{TN}{TN+FP}$}
      \node[label={[]below:\seEquation}](se){\tikz{\se}};
      \node[right=5pt of se, label={[]below:\spEquation}]{\tikz{\sp}};
    \end{tikzpicture}
  }

  \caption{Evaluation metrics:
    \protect\subref{fig:evaluation:confusion_matrix} confusion matrix,
    \protect\subref{fig:evaluation:roc_axis} \ac{se} - \ac{sp}
  }
  \label{fig:CM}
\end{figure}

All the experiments are evaluated in terms of \acf{se} and \acf{sp} using the \ac{lopocv} strategy, in line with \cite{Lemaintre2015miccaiOCT}.
\ac{se} and \ac{sp} are statistics driven from the confusion matrix as depicted in Fig.\,\ref{fig:CM}.
The \ac{se} evaluates the performance of the classifier with respect to the positive class, while the \ac{sp} evaluates its performance with respect to negative class.
The use of \ac{lopocv} implies that at each round, a pair \ac{dme}-normal volume is selected for testing while the remaining volumes are used for training.
Subsequently, no \ac{se} or \ac{sp} variance can be reported.
However, \ac{lopocv} strategy has been adopted despite this limitation due to the reduced size of the dataset.

\subsection{Experiment \#1}\label{subsec:exp1}
% Experiment structure
%
% Intro:
%   - background
%   - goal / experiment intention / why
%   - data
%   - evaluation
%   - reference to result table
%
% Procedure (by data if more than one):
%   - pre-processing
%   - feature extraction
%   - mapping
%   - feature representation
%   - classifier
%
% Remarks (if any)
%
% Result highlights:
%   - (only a description)

%% Experiment intro
This experiment intends to find the optimal number of words and its effect on the different configurations (i.e., pre-processing and feature representation), on the contrary to~\cite{Lemaintre2015miccaiOCT}, where the codebook size was arbitrarily set to $k = 32$.

%(pre-processing which consists of denoising, flattening, and aligning along different mapping).
%In order to determine the optimal size of the codebook when using \bow, this experiment evaluates several codebook sizes on SERI dataset.

%% Experiment procedure
Several pre-processing strategies are used: (i) \ac{nlm}, (ii) a combination of \ac{nlm} and flattening (\ac{nlm}+\f), and (iii) a combination of \ac{nlm}, flattening, and aligning (\ac{nlm}+\fal).
\lbp and \lbptop descriptors are detected using the default configuration.
Volumes are represented using \ac{bow}, where the codebook size ranging for $k\in \{10, 20, 30, \cdots, 100, 200, \cdots, 500,$ $1000\}$.
Finally, the volumes are classified using \lr.
The choice of this linear classifier avoids that the results get boosted by the classifier.
In this manner, any improvement would be linked to the pre-processing and the size of the codebook.
%
\begin{landscape}
  \begin{table}
\caption{Experiment \#1 - Optimum number of words for each configuration as a result of \ac{lr} Classification, for high-level feature extraction of \emph{global} and \emph{local}-\ac{lbp}, and \emph{local}-\ac{lbptop} features with different pre-processing. The pre-processing includes: \ac{nf}, \ac{f}, and \ac{fal}.
The achieved performances are indicated in terms of  \acs{acc}, \acs{f1}, \acs{se}, and \acs{sp}}
\centering

\footnotesize{
\resizebox{1\linewidth}{!}{
\begin{tabular}{ll  ccccr	c	ccccr	c ccccr}
\toprule
Features & Pre-processing &    \multicolumn{5}{c}{$\{8,1\}$}  & & \multicolumn{5}{c}{$\{16,2\}$} & & \multicolumn{5}{c}{$\{24,3\}$} \\
  \cmidrule(l){3-7}  \cmidrule(l){9-13}  \cmidrule(l){15-19}
   & &  	\ac{acc}\% & \ac{f1}\% & \ac{se}\% & \ac{sp}\%  & W\# &  & \ac{acc}\% & \ac{f1}\% & \ac{se}\% & \ac{sp}\%  & W\# &  &\ac{acc}\% & \ac{f1}\% & \ac{se}\% & \ac{sp}\%  & W\# \\
\midrule
\\[-2ex]
  	\emph{global}-\ac{lbp}		\\
 	& \acs{nf} & 81.2 &  78.5 & 68.7 &  93.7 & 500 & & 62.5 & 58.0 & 56.2 & 62.5 & 80  & & 62.5  & 62.5 & 62.5 & 62.5 & 80  \\
	& \acs{f}  & 71.9 &  71.0 & 68.7 &  75.0 & 400 & & 68.7 & 66.7 & 62.5 & 75.0 & 300 & & 68.7  & 66.7 & 62.5  & 75.0 & 300	 \\
	& \acs{fal}& 71.9 &  71.0 & 68.7 &  75.0 & 500 & & 71.9 & 71.0 & 68.7 & 75.0 & 200 & & 75.0  & 68.7 & 68.7  & 68.7 & 500	 \\
	%& \acs{fac}& 75.0 & 73.3 & 68.7 &  81.2 & 500 & & 78.1 & 75.8 & 68.7 & 87.5 & 500 & & 68.7  & 68.7 & 68.7  & 68.7 & 90	 \\
	\\
\hdashline \noalign{\vskip 3pt}
\\[-2ex]
 	\emph{local}-\ac{lbp}		\\
 	& \acs{nf}  & 75.0  & 75.0 & 75.0  & 75.0 & 70 & & 65.6 & 64.5 & 62.5 & 68.7 & 90 & &  62.5 & 60.0 & 56.2  & 68.7  & 30  \\
	& \acs{f}   & 75.0  & 73.3 & 68.7  & 81.2 & 30 & & 71.8 & 61.0 & 68.7 & 75.0 & 70 & &  62.5 & 62.5 & 62.5  & 62.5  & 100	 \\
	& \acs{fal} & 75.0  & 69.0 & 62.5  & 81.2 & 40 & & 71.9 & 71.0 & 68.7 & 75.0 & 200 & &  68.7 & 66.7 & 68.7 & 62.5 & 10	 \\
	%& \acs{fac}& 68.7 & 68.7 & 68.7 & 68.7 & 300 & & 65.6 & 64.5 & 62.5 & 68.7 & 100 & & 65.6  & 64.5 & 62.5  & 68.7 & 100	 \\
	\\
\hdashline \noalign{\vskip 3pt}
\\[-2ex]
 	\emph{local}-\ac{lbptop}		\\
 	& \acs{nf}	& 68.7 & 68.7 & 68.7 & 68.7 & 400 & & 75.0  & 75.0   & 75.0   & 75.0  & 500 & & 71.9 & 71.0 & 68.7 & 75.0 & 60	 \\
	& \acs{f}	& 68.7 & 68.7 & 68.7 & 68.7 & 300 & & 68.7  & 66.7   & 62.5   & 75.0  & 50  & & 75.0 & 76.5 & 81.2 & 68.7 & 80	 \\
	& \acs{fal}	& 75.0 & 73.3 & 68.7 & 81.2 & 100 & & 75.0  & 73.3   & 68.7   & 81.2  & 90  & & 75.0 & 69.0 & 62.5 & 81.2 & 70	 \\
	%& \acs{fac}	& 71.9 & 69.0 & 62.5 & 81.2 & 400 & & 75.0  & 73.3   & 68.7   & 81.2  & 100 & & 75.0 & 73.3 & 68.7 & 81.2 & 60	 \\
	\\

\bottomrule
\end{tabular}}}
\label{tab:table2}
\end{table}
\end{landscape}



The usual build of the codebook consists of clustering the samples in the feature space using $k$-means (see Sect.\,\ref{subsec:fearep}).
However, this operation is rather computationally expensive and the convergence of the $k$-means algorithm for all codebook sizes is not granted.
Nonetheless, Nowak\,\textit{et al.}~\cite{nowak2006sampling} pointed out that randomly generated codebooks can be used at the expenses of accuracy.
Thus, the codebook are randomly generated since the final aim is to asses the influence of the codebook size and not the performance of the framework.
For this experiment, the codebook building is carried out using random initialization using $k$-means++ algorithm~\cite{arthur2007k}, which is usually used as a $k$-means initialization algorithm.

For this experiment, \ac{se} and \ac{sp} are complemented with \ac{acc} and \ac{f1} score (see Eq.\,\eqref{eq:accf1}).
\ac{acc} offers an overall sense of the classifier performance, and \ac{f1} illustrates the trade off between \ac{se} and precision.
Precision or positive predictive value is a measure of algorithm exactness and is defined as a ratio of \ac{tp} over the total predicted positive samples.
\begin{align}
\ac{acc} = \frac{TP+TN}{TP+TN+FP+FN} \qquad \ac{f1} = \frac{2TP}{2TP +FP+FN}
\label{eq:accf1}
\end{align}
Appendix~\ref{app:1}~-~Table~\ref{tab:table2} shows the results obtained for the optimal dictionary size while the complete set of all \ac{acc} and \ac{f1} graphics can be found at~\cite{Lemaitre2015}.
According to the obtained results, it is observed that optimum number of words is smaller for \emph{local}-\ac{lbp} features in comparison to \emph{local}-\ac{lbptop} and \emph{global}-\ac{lbp}, respectively.
Using \ac{lr} classifier, the best performances were achieved using \emph{local}-\ac{lbp} with 70 words (\ac{se} and \ac{sp} of 75.0\%) and \emph{local}-\ac{lbptop} with 500 words (\ac{se} and \ac{sp} of 75.0\% as well).
These results are highlighted in Appendix~\ref{app:1}~-~Table~\ref{tab:table2}.





%In order to illustrate the impact of the dictionary size, Fig.\,\ref{fig:RBOW} illustrates the \ac{acc} and \ac{f1} graph for a particular case.
%In this figure, the classification performance of \emph{local}-\ac{lbp} features extracted from the \nlm+\f configuration is illustrated. 
%\begin{figure}
%\centering
%\includegraphics[width=0.85\textwidth]{figure2}
%\caption{The performance of \ac{lr} with NLM+\ac{f} pre-processing for different $P$ and $R$.}
%\label{fig:RBOW}
%\end{figure}
%%% Experiment Result description
%The obtained results show that commonly less number of words is required when higher number of sampling points and radius ($\{P,R\} = \{24,3\}$) are used.
%The required number of words decreases for \emph{local}-\ac{lbp} in comparison with \emph{global}-\ac{lbp}.
%Although it was expected that the use of different pre-processing steps affect the optimal number of words, this influence is not substantial nor consistent over all the obtained results.



\subsection{Experiment \#2}\label{subsec:exp2}
This experiments explores the improvement associated with: (i) different pre-processing methods and (ii) using larger range of classifiers (i.e., linear and non-linear) on the high-level representation.

All the pre-processing stages are evaluated (NLM, NLM+\acs{f}, and NLM+\acs{fal}).
In this experiment, the codebooks for the \ac{bow} representation of \ac{lbp} and \ac{lbptop} features are computed using regular $k$-means algorithm which is initialized using $k$-means++, where $k$ is chosen according to the findings of \emph{Experiment~\#1}.
Finally, the volumes are classified using $k$-\ac{nn}, \ac{rf}, \ac{gb}, and \ac{svm}.
The $k$-\ac{nn} classifier is used in conjunction with the 3 nearest-neighbors rule to classify the test set.
The \ac{rf} and \ac{gb} classifier are trained using 100 un-pruned trees, while \ac{svm} classifier is trained using an \ac{rbf} kernel and its parameters $C$, and $\gamma$ are optimized through grid-search.

Complete list of the obtained results from this experiment are shown in Appendix~\ref{app:1}~-~Table~\ref{tab:table3}.
Despite that highest performances are achieved when NLM+\acs{f} or NLM+\acs{fal} are used, most configurations decline when applied with extra pre-processing stages.
The best results are achieved using \ac{svm} followed by \ac{rf}.

\subsection{Experiment \#3}\label{subsec:exp3}
This experiment replicates the \emph{Experiment \#2} for the case of low-level representation of \ac{lbp} and \ac{lbptop} features extracted using \emph{global}-mapping.

The obtained results from this experiment are listed in Appendix~\ref{app:1}~-~Table~\ref{tab:table4}.
In this experiment, flattening the B-scan boosts the results of the best performing configuration.
However, its effects is not consistent across all the configurations.
\ac{rf} has a better performance by achieving better \ac{se} (81.2\%, 75.0\%, 68.7\%), while \ac{svm} achieve the highest \ac{sp} (93.7\%), see Appendix~\ref{app:1}~-~Table~\ref{tab:table4}.

In terms of classifier, \ac{rf} has a better performance than the others despite the fact that the highest \ac{sp} is achieved using \ac{svm}.


%%% Local Variables:
%%% mode: latex
%%% TeX-master: "../../main.tex"
%%% End:
