\begin{landscape}
  \begin{table}[ht]
\caption{Summary of all the results. The best results for each experiment are denoted in bold.}
\medskip
\scriptsize{
\begin{center}
\rowcolors{3}{black!8}{white}
\resizebox{1\linewidth}{!}{
%\rowcolors{3}{black!5}{white}
\begin{tabular}{c lr c lccc c	c c c}
\hiderowcolors
\toprule

 % \multicolumn{2}{c}{Evaluation} & \multicolumn{3}{c}{Pre-processing} & \multicolumn{5}{c}{Feature} & \multicolumn{2}{c}{Mapping} & \multicolumn{2}{c}{Feature} & Classifier & \# words\\
 % &  &  & & & \multicolumn{5}{c}{detection}  &  & \multicolumn{2}{c}{extraction} &  & \\
Experiment & \multicolumn{2}{c}{Evaluation} & Pre-processing & \multicolumn{4}{c}{Feat. Detection} & Mapping & Feat. Extraction& Classifier & \#words \\
\cmidrule(l){2-3} \cmidrule(l){5-8}
& \ac{se} & \ac{sp} &  & Type & $\{8,1\}$ & $\{16,2\}$ & $\{24,3\}$ & & & & \\
% \midrule 
% \acs{se} & \acs{sp} & \acs{nf} & \acs{f} & \acs{fal} & $\{8,1\}$ & $\{16,2\}$ & $\{24,3\}$ & \lbp & \lbptop & \emph{global} & \emph{local} & Low & High & &  \\
\midrule
% se & sp   & nf           & f            & fa           & 1            & 2            & 3            & lpb          & lbptop       & g            & l            & L            & H            & class       & words \\

\showrowcolors

%             & se   & sp   & nf      & lbp(top) & 1            & 2            & 3            & map    & f.extract. & class       & words \\
\textbf{\#3} & \textbf{81.2} & \textbf{93.7 }& \textbf{NLM+F}   & \textbf{\lbp }   & \textbf{$\checkmark$} & & & \textbf{local}  & \textbf{High }& \textbf{\ac{svm}}    & \textbf{30} \\

             \#3 & 75.0 & 93.7 & NLM+F+A & \lbp    & $\checkmark$ &              &              & local  & High & \ac{svm}    & 40 \\
             \#3 & 75.0 & 93.7 & NLM     & \lbp    & $\checkmark$ &              &              & local  & High & \ac{svm}    & 70\\
             \#3 & 75.0 & 100  & NLM     & \lbptop &              & $\checkmark$ &              & local  & High & \ac{svm}    & 500\\
             \#3 & 81.2 & 87.5 & NLM     & \lbptop & $\checkmark$ &              &              & local  & High & \ac{svm}    & 400\\
             \#3 & 81.2 & 87.5 & NLM+F+A & \lbptop &              & $\checkmark$ &              & local  & High & \ac{rf}     & 90 \\
             \#3 & 81.2 & 81.2 & NLM     & \lbp    & $\checkmark$ &              &              & local  & High & \ac{rf}     & 70 \\

\textbf{\#4}& \textbf{81.2}& \textbf{81.2}& \textbf{NLM}& \textbf{\lbptop}& & & \textbf{$\checkmark$}& \textbf{global}& \textbf{Low}& \textbf{\rf}& \\

             \#3 & 81.2 & 81.2 & NLM+F   & \lbptop & $\checkmark$ &              &              & local  & High & \ac{svm}    & 300\\
             \#4 & 81.2 & 81.2 & NLM+F+A & \lbptop &              &              & $\checkmark$ & global & Low  & \gb         & \\
             \#4 & 81.2 & 81.2 & NLM+F   & \lbptop &              &              & $\checkmark$ & global & Low  & \rf         & \\
             \#3 & 75.0 & 87.5 & NLM     & \lbp    & $\checkmark$ &              &              & local  & High & $k$-\ac{nn} & 70 \\

\textbf{\#1}&\textbf{75.0}&\textbf{87.5}&\textbf{NLM}&\textbf{\lbp}&\textbf{$\checkmark$}&&&\textbf{local}&\textbf{High}&\textbf{\rf}&\textbf{32}\\

\textbf{\#1}&\textbf{75.0}&\textbf{87.5}&\textbf{NLM}&\textbf{\lbptop}&&\textbf{$\checkmark$}&&\textbf{global}&\textbf{Low}&\textbf{\rf}&\\

             \#3 & 68.7 & 93.7 & NLM     & \lbp    & $\checkmark$ &              &              & global & High & \ac{rf}     & 500\\
             \#4 & 75   & 81.2 & NLM+F+A & \lbptop &              &              & $\checkmark$ & global & Low  & \rf         & \\
             \#3 & 68.7 & 81.2 & NLM     & \lbptop &              & $\checkmark$ &              & local  & High & \ac{rf}     & 500 \\
             \#4 & 62.5 & 93.7 & NLM     & \lbptop &              & $\checkmark$ &              & global & Low  & \svm        & \\
             \#4 & 68.7 & 87.5 & NLM     & \lbptop &              & $\checkmark$ &              & global & Low  & \rf         & \\
             \#4 & 68.7 & 81.2 & NLM     & \lbptop &              &              &              & global & Low  & \rf         & \\
             \#4 & 75.0 & 75.0 & NLM     & \lbptop &              &              &              & global & Low  & \rf         & \\
             \#4 & 68.7 & 75.0 & NLM+F   & \lbptop & $\checkmark$ &              &              & global & Low  & \svm        & \\
             \#4 & 56.2 & 75.0 & NLM     & \lbp    &              &              & $\checkmark$ & global & Low  & \rf         & \\
             \#4 & 56.2 & 75.0 & NLM+F   & \lbp    &              & $\checkmark$ &              & global & Low  & $k$-NN      & \\
             \#4 & 56.2 & 75.0 & NLM+F+A & \lbp    &              & $\checkmark$ &              & global & Low  & $k$-NN      & \\
\textbf{Venhuizen\,\emph{et al.}\,\cite{Venhuizen2015}}&\textbf{61.5}&\textbf{58.8}& \\

\bottomrule


\end{tabular}}
\end{center}}
\label{tab:results_summary}
\end{table}
\end{landscape}

\section{Results and discussion}
\label{sec:res-dis}

Table~\ref{tab:results_summary} combines the obtained results from Sect.\,\ref{sec:exp} with those reported by Lema\^{i}tre~\emph{et al.}~\cite{Lemaintre2015miccaiOCT}, while detailing the frameworks configurations.
This table shows the achieved performances with \ac{se} higher than 55\%.
% Using this last one as cut-off.
%The related experiment, choice of pre-processing, type and configuration of the features, choice of mapping and representation, choice of classifier, and finally if required the use of \ac{bow} is illustrated in this table.
%This table illustrates the highest 26 performances from highest to lowest considering their achieved \ac{se} and \ac{sp}.

%finally if required size of codebook is illustrated in this table.
The obtained results indicate that expansion and tuning of our previous framework improves the results.
%Only tuning the codebook size, based on the findings in \emph{Experiment~\#1}, leads to an improvement of 6\% in terms of \ac{se} (comparison of line 7 and line 13).
Tuning the codebook size, based on the finding of \emph{Experiment~\#1}, leads to an improvement of 6\% in terms of \ac{se} (see Table~\ref{tab:results_summary} at line 7 and 13).
Furthermore, the fine tuning of our framework (see Sect.\,\ref{sec:method}) also leads to an improvement of 6\% in both \ac{se} and \ac{sp} (see Table~\ref{tab:results_summary} at line 1 and 13).
Our framework also outperforms the proposed method of~\cite{Venhuizen2015} with an improvement of 20\% and 36\% in terms of \ac{se} and \ac{sp}, respectively.

Note that although the effects of pre-processing are not consistent through all the performance, the best results are achieved with \nlm+\f and \nlm+\fal configurations as pre-processing stages.
In general, the configurations presented in \emph{Experiment~\#2} outperform the others, in particular the high-level representation of locally mapped features with an \ac{svm} classifier.  
Focusing on the most desirable radius and sampling point configuration, smaller radius and sampling points are more effective in conjunction with local mapping, while global mapping benefit from larger radius and sampling points.

%Analyzing the obtained results, it is clear that just optimizing the codebook size without additional pre-processing step improves the results (compare line 7 (experiment~\#2) and line 13 (baseline)).

%The obtained results indicate that the highest results are achieved while flattening and flattening and alignment are added in the pre-processing step using high representation of locally mapped \ac{lbp} features (compare the first five lines).

%These results also outperform the baseline.
%In general with respect to the highest 10 performances (all outperforming the baseline), high representation of locally mapped features with \ac{svm} classifier outperform other configurations.

