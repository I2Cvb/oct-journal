\begin{landscape}
  \begin{table}[ht]
\caption{Summary of all the results. The best results for each experiment are denoted in bold.}
\medskip
\footnotesize{
\begin{center}
\rowcolors{3}{black!8}{white}
\resizebox{1\linewidth}{!}{
%\rowcolors{3}{black!5}{white}
\begin{tabular}{l c lr c lccc c	c c c}
\hiderowcolors
\toprule

 % \multicolumn{2}{c}{Evaluation} & \multicolumn{3}{c}{Pre-processing} & \multicolumn{5}{c}{Feature} & \multicolumn{2}{c}{Mapping} & \multicolumn{2}{c}{Feature} & Classifier & \# words\\
 % &  &  & & & \multicolumn{5}{c}{detection}  &  & \multicolumn{2}{c}{extraction} &  & \\
Line & Experiment & \multicolumn{2}{c}{Evaluation} & Pre-processing & \multicolumn{4}{c}{Feat. Detection} & Mapping & Feat. Representation & Classifier & \acs{bow} \\%\#words
\cmidrule(l){3-4} \cmidrule(l){6-9}
& &  \ac{se} & \ac{sp} &  & Type & $\{8,1\}$ & $\{16,2\}$ & $\{24,3\}$ & & & & \\
% \midrule 
% \acs{se} & \acs{sp} & \acs{nf} & \acs{f} & \acs{fal} & $\{8,1\}$ & $\{16,2\}$ & $\{24,3\}$ & \lbp & \lbptop & \emph{global} & \emph{local} & Low & High & &  \\
\midrule
% se & sp   & nf           & f            & fa           & 1            & 2            & 3            & lpb          & lbptop       & g            & l            & L            & H            & class       & words \\

\showrowcolors

%             & se   & sp   & nf      & lbp(top) & 1            & 2            & 3            & map    & f.extract. & class       & words \\
			 1  & \#2 & 81.2  & 93.7 & NLM+F  & \lbp   & $\checkmark$ & & & local  & High &  \ac{svm}  & $\checkmark$\\%30
             2  & \#2 & 75.0 & 93.7 & NLM+F+A & \lbp    & $\checkmark$ &              &              & local  & High & \ac{svm}    & $\checkmark$ \\%40
             3  & \#2 & 75.0 & 93.7 & NLM     & \lbp    & $\checkmark$ &              &              & local  & High & \ac{svm}    & $\checkmark$\\ %70
             4  & \#2 & 75.0 & 100  & NLM     & \lbptop &              & $\checkmark$ &              & local  & High & \ac{svm}    & $\checkmark$\\% 500
             5  & \#2 & 81.2 & 87.5 & NLM     & \lbptop & $\checkmark$ &              &              & local  & High & \ac{svm}    & $\checkmark$\\% 400
             6  & \#2 & 81.2 & 87.5 & NLM+F+A & \lbptop &              & $\checkmark$ &              & local  & High & \ac{rf}     & $\checkmark$ \\%90
             7  & \#2 & 81.2 & 81.2 & NLM     & \lbp    & $\checkmark$ &              &              & local  & High & \ac{rf}     & $\checkmark$ \\%70
			 8  & \#3 & 81.2 & 81.2 & NLM		& \lbptop & 				 & 			   & $\checkmark$  & global & Low & \ac{rf}		& \\
             9  & \#2 & 81.2 & 81.2 & NLM+F   & \lbptop & $\checkmark$ &              &              & local  & High & \ac{svm}    & $\checkmark$\\%300
             10 & \#3 & 81.2 & 81.2 & NLM+F+A & \lbptop &              &              & $\checkmark$ & global & Low  & \gb         & \\
             11 & \#3 & 81.2 & 81.2 & NLM+F   & \lbptop &              &              & $\checkmark$ & global & Low  & \rf         & \\
             12 & \#2 & 75.0 & 87.5 & NLM     & \lbp    & $\checkmark$ &              &              & local  & High & $k$-\ac{nn} & $\checkmark$ \\%70
			 13 & Lemaitre\,\emph{et al.}\,\cite{Lemaintre2015miccaiOCT} & 75.0 & 87.5 & NLM & \lbp & $\checkmark$ &&& local & High & \rf &$\checkmark$ \\% 32
			 14 & Lemaitre\,\emph{et al.}\,\cite{Lemaintre2015miccaiOCT} & 75.0 & 87.5 & NLM & \lbptop && $\checkmark$ && global & Low & \rf &\\
             15 & \#2 & 68.7 & 93.7 & NLM     & \lbp    & $\checkmark$ &              &              & global & High & \ac{rf}     & $\checkmark$\\%500
             16 & \#3 & 75   & 81.2 & NLM+F+A & \lbptop &              &              & $\checkmark$ & global & Low  & \rf         & \\
             17 & \#2 & 68.7 & 81.2 & NLM     & \lbptop &              & $\checkmark$ &              & local  & High & \ac{rf}     & $\checkmark$ \\%500
             18 & \#3 & 62.5 & 93.7 & NLM     & \lbptop &              & $\checkmark$ &              & global & Low  & \svm        & \\
             19 & \#3 & 68.7 & 87.5 & NLM     & \lbptop &              & $\checkmark$ &              & global & Low  & \rf         & \\
             20 & \#3 & 68.7 & 81.2 & NLM     & \lbptop &              &              &              & global & Low  & \rf         & \\
             21 & \#3 & 75.0 & 75.0 & NLM     & \lbptop &              &              &              & global & Low  & \rf         & \\
             22 & \#3 & 68.7 & 75.0 & NLM+F   & \lbptop & $\checkmark$ &              &              & global & Low  & \svm        & \\
             23 & \#3 & 56.2 & 75.0 & NLM     & \lbp    &              &              & $\checkmark$ & global & Low  & \rf         & \\
             24 & \#3 & 56.2 & 75.0 & NLM+F   & \lbp    &              & $\checkmark$ &              & global & Low  & $k$-NN      & \\
             25 & \#3 & 56.2 & 75.0 & NLM+F+A & \lbp    &              & $\checkmark$ &              & global & Low  & $k$-NN      & \\
			 26 & Venhuizen\,\emph{et al.}\,\cite{Venhuizen2015}& 61.5 & 58.8 & \\

\bottomrule


\end{tabular}}
\end{center}}
\label{tab:results_summary}
\end{table}
\end{landscape}

\section{Results and discussion}
\label{sec:res-dis}

Table~\ref{tab:results_summary} combines the obtained results from Sect.\,\ref{sec:exp} with those reported by Lema\^{i}tre~\emph{et al.}~\cite{Lemaintre2015miccaiOCT}, while detailing the frameworks configurations.
This table shows the achieved performances with \ac{se} higher than 55\%.

The obtained results indicate that expansion and tuning of our previous framework improves the results.
Tuning the codebook size, based on the finding of \emph{Experiment~\#1}, leads to an improvement of 6\% in terms of \ac{se} (see Table~\ref{tab:results_summary} at line 7 and 13).
Furthermore, the fine tuning of our framework (see Sect.\,\ref{sec:method}) also leads to an improvement of 6\% in both \ac{se} and \ac{sp} (see Table~\ref{tab:results_summary} at line 1 and 13).
Our framework also outperforms the proposed method of~\cite{Venhuizen2015} with an improvement of 20\% and 36\% in terms of \ac{se} and \ac{sp}, respectively.

Note that although the effects of pre-processing are not consistent through all the performance, the best results are achieved with \nlm+\f and \nlm+\fal configurations as pre-processing stages.
In general, the configurations presented in \emph{Experiment~\#2} outperform the others, in particular the high-level representation of locally mapped features with an \ac{svm} classifier.  
Focusing on the most desirable radius and sampling point configuration, smaller radius and sampling points are more effective in conjunction with local mapping, while global mapping benefit from larger radius and sampling points.
