\begin{landscape}
  \begin{table}[ht]
\caption{Summary of all the results. The best results for each experiment are denoted in bold.}
\medskip
\footnotesize{
\begin{center}
\rowcolors{3}{black!8}{white}
\resizebox{1\linewidth}{!}{
%\rowcolors{3}{black!5}{white}
\begin{tabular}{l c lr c lccc c	c c c}
\hiderowcolors
\toprule

 % \multicolumn{2}{c}{Evaluation} & \multicolumn{3}{c}{Pre-processing} & \multicolumn{5}{c}{Feature} & \multicolumn{2}{c}{Mapping} & \multicolumn{2}{c}{Feature} & Classifier & \# words\\
 % &  &  & & & \multicolumn{5}{c}{detection}  &  & \multicolumn{2}{c}{extraction} &  & \\
Line & Experiment & \multicolumn{2}{c}{Evaluation} & Pre-processing & \multicolumn{4}{c}{Feat. Detection} & Mapping & Feat. Representation & Classifier & \acs{bow} \\%\#words
\cmidrule(l){3-4} \cmidrule(l){6-9}
& &  \ac{se} & \ac{sp} &  & Type & $\{8,1\}$ & $\{16,2\}$ & $\{24,3\}$ & & & & \\
% \midrule 
% \acs{se} & \acs{sp} & \acs{nf} & \acs{f} & \acs{fal} & $\{8,1\}$ & $\{16,2\}$ & $\{24,3\}$ & \lbp & \lbptop & \emph{global} & \emph{local} & Low & High & &  \\
\midrule
% se & sp   & nf           & f            & fa           & 1            & 2            & 3            & lpb          & lbptop       & g            & l            & L            & H            & class       & words \\

\showrowcolors

%             & se   & sp   & nf      & lbp(top) & 1            & 2            & 3            & map    & f.extract. & class       & words \\
			 1  & \#2 & 81.2  & 93.7 & NLM+F  & \lbp   & $\checkmark$ & & & local  & High &  \ac{svm}  & $\checkmark$\\%30
             2  & \#2 & 75.0 & 93.7 & NLM+F+A & \lbp    & $\checkmark$ &              &              & local  & High & \ac{svm}    & $\checkmark$ \\%40
             3  & \#2 & 75.0 & 93.7 & NLM     & \lbp    & $\checkmark$ &              &              & local  & High & \ac{svm}    & $\checkmark$\\ %70
             4  & \#2 & 75.0 & 100  & NLM     & \lbptop &              & $\checkmark$ &              & local  & High & \ac{svm}    & $\checkmark$\\% 500
             5  & \#2 & 81.2 & 87.5 & NLM     & \lbptop & $\checkmark$ &              &              & local  & High & \ac{svm}    & $\checkmark$\\% 400
             6  & \#2 & 81.2 & 87.5 & NLM+F+A & \lbptop &              & $\checkmark$ &              & local  & High & \ac{rf}     & $\checkmark$ \\%90
             7  & \#2 & 81.2 & 81.2 & NLM     & \lbp    & $\checkmark$ &              &              & local  & High & \ac{rf}     & $\checkmark$ \\%70
			 8  & \#3 & 81.2 & 81.2 & NLM		& \lbptop & 				 & 			   & $\checkmark$  & global & Low & \ac{rf}		& \\
             9  & \#2 & 81.2 & 81.2 & NLM+F   & \lbptop & $\checkmark$ &              &              & local  & High & \ac{svm}    & $\checkmark$\\%300
             10 & \#3 & 81.2 & 81.2 & NLM+F+A & \lbptop &              &              & $\checkmark$ & global & Low  & \gb         & \\
             11 & \#3 & 81.2 & 81.2 & NLM+F   & \lbptop &              &              & $\checkmark$ & global & Low  & \rf         & \\
             12 & \#2 & 75.0 & 87.5 & NLM     & \lbp    & $\checkmark$ &              &              & local  & High & $k$-\ac{nn} & $\checkmark$ \\%70
			 13 & Lemaitre\,\emph{et al.}\,\cite{Lemaintre2015miccaiOCT} & 75.0 & 87.5 & NLM & \lbp & $\checkmark$ &&& local & High & \rf &$\checkmark$ \\% 32
			 14 & Lemaitre\,\emph{et al.}\,\cite{Lemaintre2015miccaiOCT} & 75.0 & 87.5 & NLM & \lbptop && $\checkmark$ && global & Low & \rf &\\
             15 & \#2 & 68.7 & 93.7 & NLM     & \lbp    & $\checkmark$ &              &              & global & High & \ac{rf}     & $\checkmark$\\%500
             16 & \#3 & 75   & 81.2 & NLM+F+A & \lbptop &              &              & $\checkmark$ & global & Low  & \rf         & \\
             17 & \#2 & 68.7 & 81.2 & NLM     & \lbptop &              & $\checkmark$ &              & local  & High & \ac{rf}     & $\checkmark$ \\%500
             18 & \#3 & 62.5 & 93.7 & NLM     & \lbptop &              & $\checkmark$ &              & global & Low  & \svm        & \\
             19 & \#3 & 68.7 & 87.5 & NLM     & \lbptop &              & $\checkmark$ &              & global & Low  & \rf         & \\
             20 & \#3 & 68.7 & 81.2 & NLM     & \lbptop &              &              &              & global & Low  & \rf         & \\
             21 & \#3 & 75.0 & 75.0 & NLM     & \lbptop &              &              &              & global & Low  & \rf         & \\
             22 & \#3 & 68.7 & 75.0 & NLM+F   & \lbptop & $\checkmark$ &              &              & global & Low  & \svm        & \\
             23 & \#3 & 56.2 & 75.0 & NLM     & \lbp    &              &              & $\checkmark$ & global & Low  & \rf         & \\
             24 & \#3 & 56.2 & 75.0 & NLM+F   & \lbp    &              & $\checkmark$ &              & global & Low  & $k$-NN      & \\
             25 & \#3 & 56.2 & 75.0 & NLM+F+A & \lbp    &              & $\checkmark$ &              & global & Low  & $k$-NN      & \\
			 26 & Venhuizen\,\emph{et al.}\,\cite{Venhuizen2015}& 61.5 & 58.8 & \\

\bottomrule


\end{tabular}}
\end{center}}
\label{tab:results_summary}
\end{table}
\end{landscape}

\section{Results and Discussion}
\label{sec:res-dis}
This section summarizes the results obtained from Sect.\,\ref{sec:exp} (extensive results can be found in Appe	ndix~\ref{app:1}) and extends the discussion.
Table~\ref{tab:results_summary} combines the obtained results from experiment~\#2~and~\#3 with those reported in Lemaitre~\emph{et al.}~\cite{Lemaintre2015miccaiOCT}.
% Using this last one as cut-off.
This table illustrates the highest 26 performances from highest to lowest considering their achieved \ac{se} and \ac{sp}.
The related experiment, choice of pre-processing, type and configuration of the features, choice of mapping and representation, choice of classifier, and finally if required the use of \ac{bow} is illustrated in this table.
%finally if required size of codebook is illustrated in this table.
Analyzing the obtained results, it is clear that just optimizing the codebook size without additional pre-processing step improves the results (compare line 7 (experiment~\#2) and line 13 (baseline)).
The obtained results indicate that the highest results are achieved while flattening and flattening and alignment are added in the pre-processing step using high representation of locally mapped \ac{lbp} features (compare the first five lines).
These results also outperform the baseline.
In general with respect to the highest 10 performances (all outperforming the baseline), high representation of locally mapped features with \ac{svm} classifier outperform other configurations.
Considering the desirable radius and sampling points it is concluded that smaller radius and sampling points is effective for local mapping while global mapping benefits from larger radius and sampling points.