\subsection{Experiment \#3}\label{subsec:exp3}
% Experiment structure
%
% Intro:
%   - background
%   - goal / experiment intention / why
%   - data
%   - evaluation
%   - reference to result table
%
% Procedure (by data if more than one):
%   - pre-processing
%   - feature extraction
%   - mapping
%   - feature representation
%   - classifier
%
% Remarks (if any)
%
% Result highlights:
%   - (only a description)

%% Experiment intro
This experiment replicates the \emph{Experiment \#2} for the case of low-level representation of \ac{lbp} and \ac{lbptop} features extracted using \emph{global}-mapping.

%% Experiment procedur
%The same pre-processing strategies (NLM, NLM+\acs{f}, and NLM+\acs{fal}) are investigated.
%lbp and \lbptop descriptors are detected using the default configuration.
%Volumes are represented using low-level feature representation of the \emph{global} mapping.
%Finally, the volumes are classified using $k$-\ac{nn}, \ac{rf}, \ac{gb}, and \ac{svm}, similarly to \emph{Experiment \#3}.

The obtained results from this experiment are listed in Appendix~\ref{app:1}~-~Table~\ref{tab:table4}.
In this experiment, flattening the B-scan boosts the results of the best performing configuration.
However, its effects is not consistent across all the configurations.
Random forest has a better performance by achieving better \ac{se} (81.2\%, 75.0\%, 68.7\%), while \ac{svm} achieve the highest \ac{sp} (93.7\%), see Appendix~\ref{app:1}~-~Table~\ref{tab:table4}.

In terms of classifier, \ac{rf} has a better performance than the others despite the fact that the highest \ac{sp} is achieved using \ac{svm}.


%%% Experiment Result description
%The obtained results from this experiment is listed in Table~\ref{tab:table4}.
%The most relevant configurations are shaded and the highest results are highlighted in \textbf{bold}.
%Similarly to the results reported in Sect.\,\ref{subsec:exp3}, the effect of flattening the B-scan boosts the results for the best performing configuration, but this effect is not consistent across all the configurations.
%For this experiment, \lbptop outperforms \lbp and larger $P$ and $R$ values for feature detection tends to obtain better results. 
%In terms of classifier, \rf have better performance than the others but the highest \ac{sp} is achieved using \svm.

% The highest results of this experiment, \ac{se} and \ac{sp} of 81.2\% and 81.2\%, respectively, was achieved with \ac{rf} and using \emph{global}-\ac{lbptop} features with sampling points and radius of $\{S,R\}=\{24,3\}$.
% In general, in this experiment, \emph{global}-\ac{lbptop} features have better performance in comparison to \emph{global}-\ac{lbp} features and the  classification rates improved while using a higher number of sampling points and radius ($\{S,R\}=\{24,3\}$).

% Similar to the previous experiments, although the effects of additional pre-processing steps (\ac{f} and \ac{fal}) is evident for \ac{rf} performance on $\{S,R\} = \{24,3\}$, similar to the previous experiments, this influence is not consistent for all different configurations, in terms of classifier and $\{S,R\}$.

%The best results (81.2\%\,\ac{se} and 81.2\%\,\ac{sp}) are achieved when using \nlm, flattening, \lbptop detection using $\{P,R\}= \{24,3\}$, global mapping, low-level representation, and \rf classifier.
%This result can be compared with other relevant results in Table~\ref{tab:results_summary}
%\begin{landscape}

  \begin{table}[ht]
\caption{Experiment \#3 - $k$-\ac{nn} and  \ac{svm} classification with \ac{bow} for the \emph{global} and \emph{local} \ac{lbp} and \emph{local} \ac{lbptop} features with different pre-processing. The optimum number of words were selected based on the previous experiment.}
\centering

\scriptsize{
\resizebox{0.9\linewidth}{!}{
\begin{tabular}{ll  lr	c	lr	c lr c lr	c	lr	c lr}
\toprule
& & \multicolumn{8}{c}{$k$-\ac{nn}} & & \multicolumn{8}{c}{\ac{svm}}\\
\cmidrule(l){3-10} \cmidrule(l){12-19}
Features & Pre-processing &    \multicolumn{2}{c}{$\{8,1\}$}  & & \multicolumn{2}{c}{$\{16,2\}$} & & \multicolumn{2}{c}{$\{24,3\}$}  & &   \multicolumn{2}{c}{$\{8,1\}$}  & & \multicolumn{2}{c}{$\{16,2\}$} & & \multicolumn{2}{c}{$\{24,3\}$} \\
  \cmidrule(l){3-4}  \cmidrule(l){6-7}  \cmidrule(l){9-10}   \cmidrule(l){12-13}  \cmidrule(l){15-16}  \cmidrule(l){18-19}
   & &  	\ac{se}\% & \ac{sp}\% &  & \ac{se}\% & \ac{sp}\% &  & \ac{se}\% & \ac{sp}\%  & & 	\ac{se}\% & \ac{sp}\% &  & \ac{se}\% & \ac{sp}\% &  & \ac{se}\% & \ac{sp}\% \\
\midrule
  	\emph{global}-\ac{lbp}		\\
 	& \acs{nf} & 43.7 &  93.7 &   & 43.7 & 87.5  &  & 43.7  & 62.5  &  & 68.7 & 87.5 & & 62.5 & 62.5 & & 50.0 & 56.2 \\  
	& \acs{f}  & 43.7 &  56.2 &   & 50.0 & 75.0  &  & 62.5  & 56.2  &  & 56.2 & 56.2 & & 56.2 & 75.0 & & 56.2 & 68.7  \\
	& \acs{fa} & 56.2 &  62.5 &   & 43.7 & 81.2  &  & 68.7  & 56.2  &  & 56.2 & 68.7 & & 68.7 & 68.7 & & 56.2 & 75.0	 \\
	%& \acs{fac}& 50   &  75.0 &   & 37.5 & 87.5  &  & 50.0  & 62.5 &  & 50.0 & 75.0 & & 50.0 & 75.0 & & 43.7 & 68.7 \\
\hdashline \noalign{\vskip 3pt}
 	\emph{local}-\ac{lbp}		\\
 	& \acs{nf} & \cellcolor[gray]{0.8}\textbf{75.0} & \cellcolor[gray]{0.8}\textbf{87.5} & & 50.0  & 68.7 &  &  43.7  & 43.7 & & \cellcolor[gray]{0.6}\textbf{75.0} & \cellcolor[gray]{0.6}\textbf{93.7} & & 50.0 & 75.0 & & 56.2 & 56.2    \\
	& \acs{f}  & \cellcolor[gray]{0.8}56.2 & \cellcolor[gray]{0.8}56.2 & & 50.0  & 50.0 &  & 50.0   & 43.7 & &\cellcolor[gray]{0.6}\textbf{81.2} & \cellcolor[gray]{0.6}\textbf{93.7} & & 68.7 & 68.7 & & 68.7 & 75.0 \\
	& \acs{fa} & \cellcolor[gray]{0.8}56.2 & \cellcolor[gray]{0.8}43.7 & & 50.0  & 75.0 &  & 50.0   & 62.5 & & \cellcolor[gray]{0.6}\textbf{75.0} & \cellcolor[gray]{0.6}\textbf{93.7} & & 75.0 & 68.7 & & 68.7 & 68.7  \\
	%& \acs{fac}& 37.5 & 75.0 & & 31.2 & 75.0 &  & 62.5  & 81.2 & & 62.5 & 87.5 & & 56.2 & 43.7 & & 75.0 & 56.2	 \\
\hdashline \noalign{\vskip 3pt}
 	\emph{local}-\ac{lbptop}		\\
 	& \acs{nf} & 56.2 & 75.0 & & 56.2 & 75.0 & & 62.5 & 56.2 & & \cellcolor[gray]{0.8}\textbf{81.2} & \cellcolor[gray]{0.8}\textbf{87.5} & & \cellcolor[gray]{0.8}\textbf{75.0} & \cellcolor[gray]{0.8}\textbf{100} & & 56.2 & 75.0 \\
	& \acs{f} & 62.5 & 43.7 & & 37.5 & 68.7 & & 43.7 & 62.5 & & \cellcolor[gray]{0.8}\textbf{81.2} & \cellcolor[gray]{0.8}\textbf{81.2} & & \cellcolor[gray]{0.8}75.0 & \cellcolor[gray]{0.8}68.7 & & 81.2 & 68.7		 \\
	& \acs{fal}	& 56.2 & 56.2 & & 68.7 & 50.0 & & 43.7 & 62.5 & & \cellcolor[gray]{0.8}62.5 & \cellcolor[gray]{0.8}75.0 & & \cellcolor[gray]{0.8}68.7 & \cellcolor[gray]{0.8}75.0 & & 62.5 & 81.2  \\
	%& \acs{fac} & 43.7 & 68.7 & & 68.7 & 75.0 & & 56.2 & 81.2 & & 56.2 & 87.5 & & 87.5 & 75.0 & & 62.5 & 75.0	 	 \\
\midrule
& & \multicolumn{8}{c}{\ac{rf}} & & \multicolumn{8}{c}{\ac{gb}}\\
\cmidrule(l){3-10} \cmidrule(l){12-19}
Features & Pre-processing &    \multicolumn{2}{c}{$8^{riu2}$}  & & \multicolumn{2}{c}{$16^{riu2}$} & & \multicolumn{2}{c}{$24^{riu2}$}  & &   \multicolumn{2}{c}{$8^{riu2}$}  & & \multicolumn{2}{c}{$16^{riu2}$} & & \multicolumn{2}{c}{$24^{riu2}$} \\
  \cmidrule(l){3-4}  \cmidrule(l){6-7}  \cmidrule(l){9-10}   \cmidrule(l){12-13}  \cmidrule(l){15-16}  \cmidrule(l){18-19}
   & &  	\ac{se}\% & \ac{sp}\% &  & \ac{se}\% & \ac{sp}\% &  & \ac{se}\% & \ac{sp}\%  & & 	\ac{se}\% & \ac{sp}\% &  & \ac{se}\% & \ac{sp}\% &  & \ac{se}\% & \ac{sp}\% \\
\midrule
  	\emph{global}-\ac{lbp}		\\
 	& \acs{nf} & \cellcolor[gray]{0.8}\textbf{68.7} & \cellcolor[gray]{0.8}\textbf{93.7} & & 43.7 & 62.5 & & 50.0 & 68.7  & & 56.2 & 50.0 & & 37.5 & 31.2 & & 50.0 & 43.7\\
	& \acs{f}  & \cellcolor[gray]{0.8}56.2 & \cellcolor[gray]{0.8}50.0 & & 56.2 & 75.0 & & 50.0 & 75.0  & & 50.0 & 56.2 & & 56.2 & 75.0 & & 43.7 & 62.5\\
	& \acs{fa} & \cellcolor[gray]{0.8}68.7 & \cellcolor[gray]{0.8}50.0 & & 56.2 & 62.5 & & 62.5 & 56.2  & & 56.2 & 50.0 & & 68.7 & 50.0 & & 43.7 & 75.0\\
	%& \acs{fac} & 68.7 & 68.7 & & 56.2 & 75.0 & & 50.0 & 68.7& &  50.0 & 68.7 & & 75.0 & 62.5 & & 56.2 & 81.2  \\ 
\hdashline \noalign{\vskip 3pt}
 	\emph{local}-\ac{lbp}		\\
 	& \acs{nf} &  \cellcolor[gray]{0.8}\textbf{81.2} & \cellcolor[gray]{0.8}\textbf{81.2} & & 62.5 & 56.2 & & 56.2 & 56.2 & & 75.0 & 62.5 & & 68.7 & 87.5 & & 50.0 & 75.0   \\
	& \acs{f}  &  \cellcolor[gray]{0.8}56.2 & \cellcolor[gray]{0.8}81.2 & & 62.5 & 68.7 & & 68.7 & 62.5 & & 68.7 & 75.0 & & 50.0 & 75.0 & & 50.0 & 62.5 \\
	& \acs{fa} &  \cellcolor[gray]{0.8}68.7 & \cellcolor[gray]{0.8}62.5 & & 62.6 & 68.7 & & 43.7 & 43.7 & & 56.2 & 50.0 & & 68.7 & 56.2 & & 50.0 & 50.0 \\
	%& \acs{fac}&  62.5 & 68.7 & & 68.7 & 68.7 & & 68.7 & 56.2 & &  56.2 & 50.0 & & 56.2 & 68.7 & & 62.5 & 68.7  \\
\hdashline \noalign{\vskip 3pt}
 	\emph{local}-\ac{lbptop}		\\
 	& \acs{nf} & 68.7 & 62.5 & & \cellcolor[gray]{0.6}\textbf{68.7} & \cellcolor[gray]{0.6}\textbf{81.2} & & 68.7 & 68.7 & & 37.5 & 68.7 & & 62.5 & 81.2 & & 62.5 & 50.0  \\
	& \acs{f} & 50.0 & 62.5 & & \cellcolor[gray]{0.6}62.5 & \cellcolor[gray]{0.6}62.5 & & 43.7 & 75.0 & & 50.0 & 56.2 & & 43.7 & 62.5 & & 50.0 & 62.5		 \\
	& \acs{fal}	& 50.0 & 62.5 & & \cellcolor[gray]{0.6}\textbf{81.2} & \cellcolor[gray]{0.6}\textbf{87.5} & & 50.0 & 68.7 & & 56.2 & 62.5 & & 81.2 & 68.7 & & 75.0 & 68.7  \\
	%& \acs{fac}	 & 56.2 & 81.2 & & 68.2 & 81.2 & & 68.7 & 87.5 & & 75.0 & 68.7 & & 87.5 & 75.0 & & 75.0 & 87.5 	 \\

\bottomrule
\end{tabular}}}
\label{tab:table3}
\end{table}
%\end{tiny}
\end{landscape}

%--------------------------------------------------------
%  \begin{table}[ht]
%\caption{The obtained results of experiment \#2.
%Classification results obtained from low-level representation of global \ac{lbp} and \ac{lbptop} features with different pre-processing.
%Pre-processing steps include: \ac{nf}, \ac{f}, \ac{fal}, and \ac{fac}.
%Different classifiers such as \ac{rf}, \ac{gb}, \ac{svm}, \ac{lr}, and $k$-\ac{nn} are used.
%}
%\medskip
%
%\footnotesize{
%\begin{center}
%\resizebox{1\linewidth}{!}{
%%\begin{tabularx}{1.02\linewidth}{l cc  cc cc c cc cc cc }
%\begin{tabular}{l cc  cc cc c cc cc cc }
%\toprule
%& \multicolumn{6}{c}{\ac{rf}} & &  \multicolumn{6}{c}{\ac{gb}} \\
%\cmidrule(l){2-7} \cmidrule(l){9-14}
%Features &  \multicolumn{2}{c}{$8^{riu2}$}  & \multicolumn{2}{c}{$16^{riu2}$} & \multicolumn{2}{c}{$24^{riu2}$} & &  
%   \multicolumn{2}{c}{$8^{riu2}$}  &  \multicolumn{2}{c}{$16^{riu2}$} & \multicolumn{2}{c}{$24^{riu2}$} \\
%  \cmidrule(l){2-3}  \cmidrule(l){4-5}  \cmidrule(l){6-7} \cmidrule(l){9-10}  \cmidrule(l){11-12}  \cmidrule(l){13-14}
%   &  	\ac{se}\% &  \ac{sp}\%  &  \ac{se}\% &  \ac{sp}\% & 	\ac{se}\% &  \ac{sp}\% & & 
%   \ac{se}\% &  \ac{sp}\% & \ac{se}\% &  \ac{sp}\% & \ac{se}\% &  \ac{sp}\%\\
%\midrule
%  	\emph{global}-\ac{lbp} \\
% 	\acs{nf} & 43.7 & 62.5 &   43.7 & 62.5 & 56.2 & 75   & &  43.7 & 43.7 & 43.7 & 37.5 & 37.5 & 31.25  		\\
%	\acs{f}  & 56.2 & 56.2 &   68.7 & 62.5 & 62.5 & 68.7 & &  25   & 56.2 & 50   & 43.7 & 25   & 43.7 \\
%	\acs{fa} & 65.2 & 56.2 &   50   & 50   & 56.2 & 68.7 & &  43.75& 62.5 & 62.5 & 50   & 31.2 & 31.2 \\
%	\acs{fac}& 56.2 & 62.5 &   56.2 & 62.5 & 68.7 & 56.2 & &  25   & 62.5 & 75   & 81.2 & 93.7 & 87.5\\
%
%\hdashline \noalign{\vskip 3pt}
% 	\emph{global}-\ac{lbptop}		\\
% 	\acs{nf}	 & 56.2 & 68.7 &   68.7  & 87.5 & 68.7  & 81.2 & &  68.7 & 68.7 & 75   & 50   & 56.2 & 43.7\\
%	\acs{f}	 & 56.2 & 62.5 &   81.2  & 68.7 & 81.2  & 81.2 & &  56.2 & 62.5 & 62.5 & 68.7 & 68.7 & 81.2\\
%	\acs{fal}& 68.7 & 62.5 &   75    & 68.7 & 75    & 81.2 & &  56.2 & 43.7 & 62.5 & 62.5 & 75   & 75 \\
%	\acs{fac}& 75   & 68.7 &   75    & 81.2 & 75    & 75   & &  75   & 75   & 75   & 56.2 & 81.2 & 62.5\\
%\midrule
%& \multicolumn{6}{c}{\ac{svm}} & &  \multicolumn{6}{c}{$k$-\ac{nn}} \\
%\cmidrule(l){2-7} \cmidrule(l){9-14}
%Features &  \multicolumn{2}{c}{$8^{riu2}$}  & \multicolumn{2}{c}{$16^{riu2}$} & \multicolumn{2}{c}{$24^{riu2}$} & &  
%   \multicolumn{2}{c}{$8^{riu2}$}  &  \multicolumn{2}{c}{$16^{riu2}$} & \multicolumn{2}{c}{$24^{riu2}$} \\
%  \cmidrule(l){2-3}  \cmidrule(l){4-5}  \cmidrule(l){6-7} \cmidrule(l){9-10}  \cmidrule(l){11-12}  \cmidrule(l){13-14}
%   &  	\ac{se}\% &  \ac{sp}\%  &  \ac{se}\% &  \ac{sp}\% & 	\ac{se}\% &  \ac{sp}\% & & 
%   \ac{se}\% &  \ac{sp}\% & \ac{se}\% &  \ac{sp}\% & \ac{se}\% &  \ac{sp}\%\\
%\midrule
%  	\emph{global}-\ac{lbp}		\\
% 	\acs{nf} & 56.2 & 62.5 & 56.2 & 43.7 & 56.2 & 68.7 & & 37.5 & 50   & 25   & 50   & 37.5 & 68.7 \\
%	\acs{f}  & 75   & 68.7 & 62.5 & 62.5 & 62.5 & 68.7 & & 62.5 & 50   & 56.2 & 75   & 62.5 & 68.7\\
%	\acs{fa} & 75   & 68.7 & 62.5 & 62.5 & 62.5 & 68.7 & & 56.2 & 50   & 56.2 & 75   & 62.5 & 68.7 \\
%	\acs{fac}& 56.2 & 62.5 & 25   & 50   & 43.7 & 62.5 & & 50   & 43.7 & 68.7 & 62.5 & 62.5 & 62.5\\
%
%\hdashline \noalign{\vskip 3pt}
% 	\emph{global}-\ac{lbptop}		\\
% 	\acs{nf}	  & 62.5 & 75   & 62.5 & 93.7 & 56.2 & 87.5 & & 31.2 & 93.7 & 37.5 & 100  & 37.5 & 81.2 \\
%	\acs{f}	  & 68.7 & 75   & 43.7 & 68.7 & 68.7 & 56.2 & & 50   & 56.2 & 56.2 & 75   & 56.2 & 62.5\\
%	\acs{fal} & 68.7 & 62.5 & 62.5 & 56.2 & 56.2 & 68.7 & & 75   & 43.7 & 56.2 & 43.7 & 68.7 & 50  \\
%	\acs{fac} & 68.7 & 68.7 & 68.7 & 87.5 & 68.7 & 87.5 & & 62.5 & 62.5 & 68.7 & 68.7 & 75   & 75\\
%%\midrule
%%
%%& \multicolumn{6}{c}{\ac{lr}} \\
%%\cmidrule(l){2-7} 
%%Features &  \multicolumn{2}{c}{$8^{riu2}$}  & \multicolumn{2}{c}{$16^{riu2}$} & \multicolumn{2}{c}{$24^{riu2}$}  \\
%%  \cmidrule(l){2-3}  \cmidrule(l){4-5}  \cmidrule(l){6-7} 
%%   &  	\ac{se}\% &  \ac{sp}\%  &  \ac{se}\% &  \ac{sp}\% & 	\ac{se}\% &  \ac{sp}\% \\
%%\midrule
%%
%%\emph{global}-\ac{lbp}		\\
%%	\acs{nf}	  & 56.2 & 43.7 & 43.7 & 56.2 & 50 & 68.7\\
%%	\acs{f}	  & 56.2 & 43.7 & 56.2 & 37.5 & 68.7 & 43.7 \\
%%	\acs{fal} & 56.2 & 43.7 & 56.2 & 31.2 & 68.7 & 43.7 \\
%%	\acs{fac} & 62.5 & 56.2 & 56.2 & 62.5 & 56.2 & 56.2\\
%%
%%
%%\hdashline \noalign{\vskip 3pt}
%% 	\emph{global}-\ac{lbptop}		\\
%% 	\acs{nf}  & 50   & 50 &  6 & 93.7 & 0 & 100	\\
%%	\acs{f}	  & 75   & 18 & 50 & 75 & 25 & 81.2\\
%%	\acs{fal} & 93.7 & 6  & 93.7 & 6 & 50 & 43.7\\
%%	\acs{fac} & 50   & 81.2 & 0 & 93.7 & 0 & 100\\
%
%\bottomrule
%
%
%\end{tabular}}
%
%\end{center}}
%\label{tab:table4}
%\end{table}
%\begin{landscape}

  \begin{table}[ht]
\caption{Experiment \#3 - Classification results obtained from low-level representation of global \ac{lbp} and \ac{lbptop} features with different pre-processing.
Pre-processing steps include: \ac{nf}, \ac{f}, \ac{fal}.
Different classifiers such as \ac{rf}, \ac{gb}, \ac{svm}, and $k$-\ac{nn} are used.
The most relevant configurations are shaded and the highest results are highlighted in \textbf{bold}.
The configurations which their performances declines with additional pre-processing are shaded in light gray while those with the opposite behavior are shaded with darker gray color.}

\medskip

\scriptsize{
\begin{center}
\resizebox{1\linewidth}{!}{
%\begin{tabularx}{1.02\linewidth}{l cc  cc cc c cc cc cc }
\begin{tabular}{ll cc  cc cc c cc cc cc }
\toprule
&  & \multicolumn{6}{c}{$k$-\ac{nn}} & &  \multicolumn{6}{c}{\ac{svm}} \\
\cmidrule(l){3-8} \cmidrule(l){10-15}
Features & Pre-processing &   \multicolumn{2}{c}{$\{8,1\}$}  & \multicolumn{2}{c}{$\{16,2\}$} & \multicolumn{2}{c}{$\{24,3\}$} & &  
   \multicolumn{2}{c}{$\{8,1\}$}  &  \multicolumn{2}{c}{$\{16,2\}$} & \multicolumn{2}{c}{$\{24,3\}$} \\
  \cmidrule(l){3-4}  \cmidrule(l){5-6}  \cmidrule(l){7-8} \cmidrule(l){10-11}  \cmidrule(l){12-13}  \cmidrule(l){14-15}
   &  & 	\ac{se}\% &  \ac{sp}\%  &  \ac{se}\% &  \ac{sp}\% & 	\ac{se}\% &  \ac{sp}\% & & 
   \ac{se}\% &  \ac{sp}\% & \ac{se}\% &  \ac{sp}\% & \ac{se}\% &  \ac{sp}\%\\
\midrule
  	\emph{global}-\ac{lbp}		\\
 	& \acs{nf} & 37.5 & 50.0   & 25.0 & 50.0   & 37.5 & 68.7 & & 56.2 & 62.5 & 56.2 & 43.7 & 56.2 & 68.7 \\
	& \acs{f}  & 62.5 & 50.0   & 56.2 & 75.0   & 62.5 & 68.7 & & 75.0 & 68.7 & 62.5 & 62.5 & 62.5 & 68.7 \\
	& \acs{fa} & 56.2 & 50.0   & 56.2 & 75.0   & 62.5 & 68.7 & & 75.0 & 68.7 & 62.5 & 62.5 & 62.5 & 68.7 \\
	%\acs{fac}& 50   & 43.7 & 68.7 & 62.5 & 62.5 & 62.5 & & 56.2 & 62.5 & 25   & 50   & 43.7 & 62.5  \\

\hdashline \noalign{\vskip 3pt}
 	\emph{global}-\ac{lbptop}		\\
 	& \acs{nf}	  &  31.2 & 93.7 & 37.5 & 100.0  & 37.5 & 81.2 & &  62.5 & 75.0   & \cellcolor[gray]{0.8}\textbf{62.5} & \cellcolor[gray]{0.8}\textbf{93.7} & 56.2 & 87.5  \\
	& \acs{f}	  &  50.0 & 56.2 & 56.2 & 75.0   & 56.2 & 62.5 & &  68.7 & 75.0   & \cellcolor[gray]{0.8}43.7 & \cellcolor[gray]{0.8}68.7 & 68.7 & 56.2 \\
	& \acs{fal}   &  75.0 & 43.7 & 56.2 & 43.7   & 68.7 & 50.0 & &  68.7 & 62.5   & \cellcolor[gray]{0.8}62.5 & \cellcolor[gray]{0.8}56.2 & 56.2 & 68.7  \\
	%\acs{fac} & 68.7 & 68.7 & 68.7 & 87.5 & 68.7 & 87.5 & & 62.5 & 62.5 & 68.7 & 68.7 & 75   & 75\\	
\midrule	
&  & \multicolumn{6}{c}{\ac{rf}} & &  \multicolumn{6}{c}{\ac{gb}} \\
\cmidrule(l){3-8} \cmidrule(l){10-15}
Features & Pre-processing &   \multicolumn{2}{c}{$8^{riu2}$}  & \multicolumn{2}{c}{$16^{riu2}$} & \multicolumn{2}{c}{$24^{riu2}$} & &  
   \multicolumn{2}{c}{$8^{riu2}$}  &  \multicolumn{2}{c}{$16^{riu2}$} & \multicolumn{2}{c}{$24^{riu2}$} \\
  \cmidrule(l){3-4}  \cmidrule(l){5-6}  \cmidrule(l){7-8} \cmidrule(l){10-11}  \cmidrule(l){12-13}  \cmidrule(l){14-15}
   & &  	\ac{se}\% &  \ac{sp}\%  &  \ac{se}\% &  \ac{sp}\% & 	\ac{se}\% &  \ac{sp}\% & & 
   \ac{se}\% &  \ac{sp}\% & \ac{se}\% &  \ac{sp}\% & \ac{se}\% &  \ac{sp}\%\\
\midrule
  	\emph{global}-\ac{lbp} \\
 	& \acs{nf} & 43.7 & 62.5 &   43.7 & 62.5 & 56.2 & 75   & &  43.7 & 43.7 & 43.7 & 37.5 & 37.5 & 31.25  		\\
	& \acs{f}  & 56.2 & 56.2 &   68.7 & 62.5 & 62.5 & 68.7 & &  25   & 56.2 & 50.0   & 43.7 & 25.0   & 43.7 \\
    & \acs{fal} & 65.2 & 56.2 &   50.0   & 50.0   & 56.2 & 68.7 & &  43.75& 62.5 & 62.5 & 50.0   & 31.2 & 31.2 \\
	%& \acs{fac}& 56.2 & 62.5 &   56.2 & 62.5 & 68.7 & 56.2 & &  25   & 62.5 & 75   & 81.2 & 93.7 & 87.5\\

\hdashline \noalign{\vskip 3pt}
 	\emph{global}-\ac{lbptop}		\\
 	& \acs{nf}	 & 56.2 & 68.7 &   \cellcolor[gray]{0.8}\textbf{68.7}  & \cellcolor[gray]{0.8}\textbf{87.5} & \cellcolor[gray]{0.6}\textbf{68.7}  & \cellcolor[gray]{0.6}\textbf{81.2} & &  68.7 & 68.7 & 75.0   & 50.0   & \cellcolor[gray]{0.8}56.2 & \cellcolor[gray]{0.8}43.7\\
	& \acs{f}	 & 56.2 & 62.5 &   \cellcolor[gray]{0.8}81.2  & \cellcolor[gray]{0.8}68.7 & \cellcolor[gray]{0.6}\textbf{81.2}  & \cellcolor[gray]{0.6}\textbf{81.2} & &  56.2 & 62.5 & 62.5 & 68.7 & \cellcolor[gray]{0.8}68.7 & \cellcolor[gray]{0.8}81.2\\
	& \acs{fal}& 68.7 & 62.5 &   \cellcolor[gray]{0.8}75.0    & \cellcolor[gray]{0.8}68.7 & \cellcolor[gray]{0.6}\textbf{75.0}   & \cellcolor[gray]{0.6}\textbf{81.2} & &  56.2 & 43.7 & 62.5 & 62.5 &\cellcolor[gray]{0.8}\textbf{75.0}   & \cellcolor[gray]{0.8}\textbf{75.0} \\
	%\acs{fac}& 75   & 68.7 &   75    & 81.2 & 75    & 75   & &  75   & 75   & 75   & 56.2 & 81.2 & 62.5\\



%\midrule
%
%& \multicolumn{6}{c}{\ac{lr}} \\
%\cmidrule(l){2-7} 
%Features &  \multicolumn{2}{c}{$8^{riu2}$}  & \multicolumn{2}{c}{$16^{riu2}$} & \multicolumn{2}{c}{$24^{riu2}$}  \\
%  \cmidrule(l){2-3}  \cmidrule(l){4-5}  \cmidrule(l){6-7} 
%   &  	\ac{se}\% &  \ac{sp}\%  &  \ac{se}\% &  \ac{sp}\% & 	\ac{se}\% &  \ac{sp}\% \\
%\midrule
%
%\emph{global}-\ac{lbp}		\\
%	\acs{nf}	  & 56.2 & 43.7 & 43.7 & 56.2 & 50 & 68.7\\
%	\acs{f}	  & 56.2 & 43.7 & 56.2 & 37.5 & 68.7 & 43.7 \\
%	\acs{fal} & 56.2 & 43.7 & 56.2 & 31.2 & 68.7 & 43.7 \\
%	\acs{fac} & 62.5 & 56.2 & 56.2 & 62.5 & 56.2 & 56.2\\
%
%
%\hdashline \noalign{\vskip 3pt}
% 	\emph{global}-\ac{lbptop}		\\
% 	\acs{nf}  & 50   & 50 &  6 & 93.7 & 0 & 100	\\
%	\acs{f}	  & 75   & 18 & 50 & 75 & 25 & 81.2\\
%	\acs{fal} & 93.7 & 6  & 93.7 & 6 & 50 & 43.7\\
%	\acs{fac} & 50   & 81.2 & 0 & 93.7 & 0 & 100\\

\bottomrule


\end{tabular}}
\end{center}}
\label{tab:table4}
\end{table}
\end{landscape}

