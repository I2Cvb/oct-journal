%% Acronym definition example using glossaries package
%% \usepackage{acro} is required
%% 
%% For a powerful usage of the acro package look at http://tex.stackexchange.com/questions/135975/how-to-define-an-acronym-by-using-other-acronym-and-print-the-abbreviations-toge

\DeclareAcronym{us}{
  short = US,
  long  = Ultra-Sound
}

\DeclareAcronym{cad}{
  short = CAD,
  long  = Computer Aided Diagnosis
}

\DeclareAcronym{dm}{
  short = DM,
  long  = Digital Mammography
}

\DeclareAcronym{gt}{
  short = GT,
  long  = Ground Truth
}

\DeclareAcronym{bus}{
%  short = B\acs*{us},
%  long  = Breast \acifused{us}{\acs*{us}}{\acl*{us}}
short = BUS,
long= Breast Ultra-Sound
}

\DeclareAcronym{ml}{
  short = ML,
  long  = Machine Learning
}

\DeclareAcronym{acm}{
  short = ACM,
  long  = Active Contour Model
}

\DeclareAcronym{crf}{
  short = CRFs,
  long  = Conditional Random Fields
}

\DeclareAcronym{mrf}{
  short = MRFs,
  long  = Markov Random Fields
}

\DeclareAcronym{cv}{
  short = CV,
  long  = Computer Vision
}
\DeclareAcronym{icm}{
  short = ICM,
  long  = Iterated Conditional Modes
}
\DeclareAcronym{sa}{
  short = SA,
  long  = Simulate Anealing
}
\DeclareAcronym{gc}{
  short = GC,
  long  = Graph-Cuts
}

\DeclareAcronym{aov}{
  short = AOV,
  long  = Area Overlap
}

\DeclareAcronym{birads}{
  short = BI-RADS,
  long  = Breast Imaging-Reporting and Data System
}

\DeclareAcronym{mad}{
  short = MAD,
  long  = Median Absolute Deviation
}

\DeclareAcronym{qc}{
  short = QC,
  long  = Quadratic-Chi
}

\DeclareAcronym{sift}{
  short = SIFT,
  long  = Self-Invariant Feature Transform
}

\DeclareAcronym{bof}{
  short = BoF,
  long  = Back-of-Features
}

\DeclareAcronym{acr}{
  short = ACR,
  long  = American College of Radiology
}

\DeclareAcronym{fa}{
  short = FA,
  long  = Fibro-Adenoma
}

\DeclareAcronym{dic}{
  short = DIC,
  long  = Ductal Inflating Carcinoma
}

\DeclareAcronym{ilc}{
  short = ILC,
  long  = Inflating Lobular Carcinoma
}

\DeclareAcronym{fpr}{
  short = FPR,
  long  = False Positive Ratio
}

\DeclareAcronym{fnr}{
  short = FNR,
  long  = False Negative Ratio
}

\DeclareAcronym{fn}{
  short = FN,
  long  = False Negative 
}

\DeclareAcronym{fp}{
  short = FP,
  long  = False Positive
}

\DeclareAcronym{rbf}{
  short = RBF,
  long  = Radial Basis Function
}

\DeclareAcronym{dr}{
  short = DR,
  long  = Diabetic Retinopathy
}

\DeclareAcronym{dme}{
  short = DME,
  long  = Diabetic Macular Edema
}

\DeclareAcronym{oct}{
  short = OCT,
  long  = Optical Coherence Tomography
}

\DeclareAcronym{sdoct}{
  short = SD-OCT,
  long  = Spectral Domain OCT
}

\DeclareAcronym{amd}{
  short = AMD,
  long = Age-related Macular Degeneration
}

\DeclareAcronym{hog}{
  short = HOG,
  long = Histogram of Oriented Gradients
}

\DeclareAcronym{svm}{
  short = SVM,
  long = Support Vector Machines
}

\DeclareAcronym{bow}{
  short = BoW,
  long = Bag-of-Words
}

\DeclareAcronym{rf}{
  short = RF,
  long = Random Forest
}

\DeclareAcronym{tp}{
	short = TP, 
	long = True Positive
}

\DeclareAcronym{tn}{
	short = TN, 
	long = True Negative
}

\DeclareAcronym{roc}{
  short = ROC,
  long = Receiver Operating Characteristic
}

\DeclareAcronym{auc}{
  short = AUC,
  long = Area Under the Curve
}

\DeclareAcronym{lbp}{
  short = LBP,
  long = Local Binary Patterns
}

\DeclareAcronym{pca}{
  short = PCA,
  long = Principal Component Analysis
}

\DeclareAcronym{nlm}{
  short = NL-means,
  long = Non-Local Means
}

\DeclareAcronym{lopocv}{
  short = LOPO-CV,
  long =  Leave-One-Patient Out Cross-Validation
}

\DeclareAcronym{lbptop}{
  short = LBP-TOP,
  long =  LBP from Three Orthogonal Planes
}

\DeclareAcronym{se}{
  short = SE,
  long =  Sensitivity
}

\DeclareAcronym{sp}{
  short = SP,
  long =  Specificity
}

\DeclareAcronym{sw}{
  short = P,
  long =  patch 
}
\DeclareAcronym{nn}{
	short = NN,
	long = Nearest Neighbor
}
\DeclareAcronym{gb}{
	short = GB,
	long = Gradient Boosting
}

\DeclareAcronym{lr}{
	short = LR,
	long = Logistic Regression 
}
\DeclareAcronym{adb}{
	short = AdB, 
	long = AdaBoost
}
\DeclareAcronym{acc}{
	short = ACC, 
	long = Accuracy
}
\DeclareAcronym{f1}{
	short = F1, 
	long = F1-score
}

\DeclareAcronym{nf}{
	short = NF, 
	long = non-flatten
}
\DeclareAcronym{f}{
	short = F, 
	long = flatten
}
\DeclareAcronym{fal}{
	short = F+A, 
	long = flatten-aligned
}
\DeclareAcronym{fac}{
	short = F+A+C, 
	long = flatten-aligned-cropped
}


\DeclareAcronym{rpe}{
  short = RPE,
  long = Retinal Pigment Epithelium
}

