%  The subcaption package allows for subfloat figure within a single float.
%  This package substitutes the depregated subfigure and subfig packages 
%  allowing to have subfigures within figures, or subtables within table 
%  floats. Subfloats have their own caption, and an optional global 
%  caption. 
%  >> WARNING: some journal templates from Springer and IEETrans might not
%              be compatible with this package forcing to use the 
%              deprecated packages instead.
% \usepackage{subcaption}
% \usepackage{subfig}
\usepackage{subfigure}

%  The following command loads a graphics package to include images 
%  in the document. It may be necessary to specify a DVI driver option,
%  e.g., [dvips], but that may be inappropriate for some LaTeX 
%  installations. 
\usepackage[]{graphicx}

% Document structure package
%
\usepackage{standalone}

% In order to include files without having a clear page using \include*, 
% the newclude package is required
\usepackage{newclude}

\usepackage{array}

% Required for acronyms
% use \acresetall to reset the acroyms counter
\usepackage[macros=true]{acro}

% Managing TODOES and unfinished figures
\usepackage{todonotes}

% Some packages useful for edition
\usepackage{changebar}
\usepackage{changes}

% Package for nice table
\usepackage{booktabs}
\usepackage{datatool}

% Mathematics extra symols and commands
\usepackage{amssymb, amsmath}
\usepackage{pifont,amsfonts} % import fonts for tick and x-mark
  % define the extra symbols
  \newcommand{\cmarkgLarge}{\text{\large \color{green!60!black!80}\ding{51}}}
  \newcommand{\cmarkrLarge}{\text{\large \color{red!60!black!80}\ding{51}}}
  \newcommand{\xmarkLarge}{\text{\large \color{red!60!black!80}\ding{55}}}
  \newcommand{\cmark}{\text{\color{green!60!black!80}\ding{51}}}
  \newcommand{\xmark}{\text{\color{red!60!black!80}\ding{55}}}

%% In order to draw some graphs
\usepackage{tikz,xifthen}
\usepackage{tikz-qtree}
\usetikzlibrary{decorations.pathmorphing} % noisy shapes
\usetikzlibrary{fit}                                            % fitting shapes to coordinates
\usetikzlibrary{backgrounds}                                    % drawing the background after the foreground
\usetikzlibrary{shapes,arrows,shadows}
\usetikzlibrary{calc,decorations.pathreplacing,decorations.markings,positioning}
\usetikzlibrary{snakes,decorations.text,shapes,patterns}
\usetikzlibrary{snakes}
\usetikzlibrary{decorations}
\usetikzlibrary{decorations.text}
\usetikzlibrary{decorations.markings}
\usetikzlibrary{shapes}
\usetikzlibrary{patterns}
\usepackage{pgfplots}

%%----- To generate stand onle tikz legends
% argument #1: any options
\newenvironment{customlegend}[1][]{%
    \begingroup
    % inits/clears the lists (which might be populated from previous
    % axes):
    \csname pgfplots@init@cleared@structures\endcsname
    \pgfplotsset{#1}%
}{%
    % draws the legend:
    \csname pgfplots@createlegend\endcsname
    \endgroup
}%
% makes \addlegendimage available (typically only available within an
% axis environment):
\def\addlegendimage{\csname pgfplots@addlegendimage\endcsname}
\pgfkeys{/pgfplots/number in legend/.style={%
        /pgfplots/legend image code/.code={%
            \node at (0.295,-0.0225){#1};
        },%
    },
}
%%---- end tikz legends


% Clever cross referencing. Using cleverref, instead of writting 
% figure~\ref{...} or equation~\ref{...}, only \cref{...} is required.
% The package interprates the references and introduces the figure, fig.,
% equation, eq., etc keywords. \Cref forces first letter capital. 
% >> WARNING: This package needs to be loaded after hyperref, math packages,
%             etc. if used.
%             Cleveref is recomended to load late
%\usepackage{hyperref}
\usepackage{cleveref}

% SI units
\usepackage{siunitx}
% Define the money way to write
\sisetup{
  group-four-digits = true,
  group-separator = {,}
}
\DeclareSIUnit\px{px}

% Nice package with citeauthor
%\usepackage{natbib}
